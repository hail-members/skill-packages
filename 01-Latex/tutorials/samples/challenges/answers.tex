\documentclass[11pt]{amsart}
% You can import at most two packages here.
\usepackage{hyperref}
\usepackage{tikz}

\theoremstyle{definition}
\newtheorem{challenge}{Challenge}

\newcommand{\N}{\mathbf{N}}
\newcommand{\R}{\mathbf{R}}
% Define \barliminf here.
\DeclareMathOperator*{\barliminf}{\underline{lim}}

\begin{document}

\title{An Intermediate Article}
\author{Your Name$\,_\circ$} % Put your name here.
% Add the address, email, and abstract here.
\address{KAIST, 291 Daehak-ro, Yuseong-gu, Daejeon 34141, Republic of Korea}
\email{your\_id@kaist.ac.kr}
\begin{abstract}
In this note, we prove that our \LaTeX\ is at an \mbox{intermediate level.}
\end{abstract}

\maketitle

Reproduce this article as closely as possible (except your name and
email address) by filling in the designated spots in the given
\texttt{tex} file.
The spots are marked with comments, and you cannot edit elsewhere.

\begin{challenge}[citing a 1656 paper]
$\binom{2n}{n} \sim 4^n/\sqrt{\pi n}$ follows from
Wallis's product formula \cite{Wallis}.
\end{challenge}

\begin{challenge}[Fatou's lemma]
% Define \barliminf in the preamble, below the definition of \R.
If $f_n \to f$ a.s., then
\[ \int f \,d\mu \le \barliminf_{n\to\infty} \int f_n \,d\mu. \]
\end{challenge}

\begin{challenge}[a summation]
\[
\sideset{}{'}\sum_{x\in A} f(x) =
\sum_{\substack{x\in A \\ x\ne 0}} f(x).
\]
\end{challenge}

\begin{challenge}[a superfluous statement]
Consider real-valued functions $f_1$,$f_2$, $\ldots\,$, all defined on
\[
R := \left]-\infty,0\right[ \times \{0,1\} \equiv
\bigl\{\,(x,y) \bigm| 0 < x < 1 \text{ and } y \in \{0,1\}\,\bigr\}.
\]
If $f_n = f_m$ a.e.\ for all $n,m \in \mathbf{N}$, then $f_n \to f$ a.e.\
for some $f$.
\end{challenge}

\begin{challenge}[the harmonic series] The harmonic series diverges.
\begin{proof}
\[
\begin{split}
\sum_{n=1}^\infty \frac{1}{n} &=
\frac{1}{1} + \underbrace{\frac{1}{2} + \frac{1}{3}}_2 +
\underbrace{\frac{1}{4} + \frac{1}{5} + \frac{1}{6} + \frac{1}{7}}_4 \\
&\quad + \underbrace{\frac{1}{8} + \frac{1}{9} + \frac{1}{10} + \frac{1}{11}
+ \frac{1}{12} + \frac{1}{13} + \frac{1}{14} + \frac{1}{15}}_8 + \cdots \\
&\ge \frac{1}{2} + \frac{1}{2} + \frac{1}{2} + \frac{1}{2} + \cdots = \infty.
\qedhere
\end{split}
\]
\end{proof}
\end{challenge}

\begin{challenge}[a symbol that doesn't exist]
$X \perp\!\!\!\perp Y$.
\end{challenge}

\begin{challenge}[Knuth's up-arrow notation]
\[
2\uparrow\uparrow k \stackrel{\mathrm{def}}{=}
2^{2^{2^{\cdot^{\cdot^{\cdot^{2}}}}}} \raisebox{7pt}{$\Big\}\scriptstyle k$}.
\]
\end{challenge}

% Define the environment tfprob and possibly some lengths and counters here.
\newlength{\tfanslen}
\settowidth{\tfanslen}{Ans: \rule{6ex}{0.2mm}}
\newlength{\tfproblen}
\setlength{\tfproblen}{\textwidth}
\addtolength{\tfproblen}{-\tfanslen}
\addtolength{\tfproblen}{-2ex}
\newcounter{tfcnt}

\newenvironment{tfprob}{%
\addtocounter{tfcnt}{1}%
\vspace{1em}\begin{minipage}[t]{\tfproblen}{\large \bfseries \thetfcnt.}}{%
\end{minipage}\hspace{2ex}Ans: \rule{6ex}{0.2mm}\vspace{1em}}

\begin{challenge}[true/false questions]
Answer the following with T or F.

\begin{tfprob}
There exists a well-order on $\prec$ on $\R$ such that for any $x \in \R$
the set $\{\,y \in \R \mid y \prec x\,\}$ is countable.
\end{tfprob}

\begin{tfprob}
Ap\'ery's constant
\[ \zeta(3) := \sum_{n=1}^\infty \frac{1}{n^3} \]
is transcendental.
\end{tfprob}

\begin{tfprob}
There exists a polynomial-time algorithm that determines if a given
graph is $3$-colorable.
\end{tfprob}
\end{challenge}

% You may add some code here.
\renewcommand{\thechallenge}{10\protect\footnote{There is no iPhone 9 or
Windows 9. Why should we have one?}}
\begin{challenge}[a popular number]
For each $n \in \N$, let $D(n)$ be the sum of all positive divisors of $n$.
What is $\sum_{n=1}^9 D(n)$?

\hfill Answer: \framebox[2cm]{\rule[-.5cm]{0pt}{1cm}}
\end{challenge}

% You may add some code here.
\renewcommand{\thechallenge}{\arabic{challenge}}
\addtocounter{challenge}{1}

\begin{challenge}[countdown]
% Define renumerate and \ritem here.
\newenvironment{renumerate}[1]{%
\begin{enumerate}
\setcounter{enumi}{#1}
\addtocounter{enumi}{1}
}{%
\end{enumerate}
}
\newcommand{\ritem}{\addtocounter{enumi}{-2}\item}

A blade is aiming for the top.
\begin{renumerate}{3}
\ritem
Three!
\ritem
Two!
\ritem
One!
\ritem
Go \ldots\ shoot!
\end{renumerate}
Counting down is not the cup of tea of the other blade.
\begin{renumerate}{2}
\ritem
Three!
\ritem
Two!
\ritem
Oh, shoot.
\end{renumerate}
\end{challenge}

\begin{challenge}[a cyclic order]
In $dx\,dy$, $dy\,dz$, and $dz\,dx$, the basic $1$-forms $dx$, $dy$, and $dz$
always appear in cyclic pairs. See Figure \ref{fig:cycle}.

\begin{figure}[h]
\begin{tikzpicture}
	\draw (90:1.5) node {$dx$};
	\draw (-30:1.5) node {$dy$};
	\draw (210:1.5) node {$dz$};
	\draw [->] (70:1.5) arc [radius=1.5, start angle=70, end angle=-10];
	\draw [->] (-50:1.5) arc [radius=1.5, start angle=-50, end angle=-130];
	\draw [->] (190:1.5) arc [radius=1.5, start angle=190, end angle=110];
\end{tikzpicture}
\caption{The cyclic order of $dx$, $dy$, and $dz$.}
\label{fig:cycle}
\end{figure}
\end{challenge}

\begin{challenge}[Riemann sums]
Some Riemann sums of $f(x):=x^2/3$ over $[0,3]$ are shown in Figure
\ref{fig:riemann}.
% Define \riemann here.
\newcommand{\riemann}[1]{%
\begin{tikzpicture}
	\foreach \k in {1,...,#1} {
		\draw [fill=lightgray] ({3*(\k-1)/#1},0) rectangle
			(3*\k/#1,{(3*(\k-1)/#1)^2/3});
	}
	\draw [->] (0,0) -- (3.5,0) node [below right] {$x$};
	\draw [->] (0,0) -- (0,3.5) node [left] {$y$};
	\draw [domain=0:3] plot (\x, {\x^2/3});
\end{tikzpicture}
}
\begin{figure}[h]
\riemann{5}
\riemann{10}

\riemann{20}
\riemann{40}
\caption{The Riemann sums for $n=5$, $10$, $20$, and $40$.}
\label{fig:riemann}
\end{figure}
\end{challenge}

% Add some code for the bibliography here.
\begin{thebibliography}{Wal56}
\bibitem[Wal56]{Wallis} Wallis, J. (1656). \textit{Arithmetica Infinitorum.} Oxford, England. Available at: archiv.org/details/ArithmeticalInfinitorum/page/n5 Available in English as \textit{The Arithmetic of Infinitesimals} (2004). (Stedall, J. A., trans.) New York, NY: Springer-Verlag.
\end{thebibliography}
\end{document}