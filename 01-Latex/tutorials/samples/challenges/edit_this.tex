\documentclass[11pt]{amsart}
% You can import at most two packages here.


\theoremstyle{definition}
\newtheorem{challenge}{Challenge}

\newcommand{\N}{\mathbf{N}}
\newcommand{\R}{\mathbf{R}}
% Define \barliminf here.

\begin{document}

\title{An Intermediate Article}
\author{Your Name$\,_\circ$} % Put your name here.
% Add the address, email, and abstract here.


\maketitle

Reproduce this article as closely as possible (except your name and
email address) by filling in the designated spots in the given
\texttt{tex} file.
The spots are marked with comments, and you cannot edit elsewhere.

\begin{challenge}[citing a 1656 paper]
$\binom{2n}{n} \sim 4^n/\sqrt{\pi n}$ follows from
Wallis's product formula \cite{Wallis}.
\end{challenge}

\begin{challenge}[Fatou's lemma]
% Define \barliminf in the preamble, below the definition of \R.
If $f_n \to f$ a.s., then
\[ \int f \,d\mu \le \barliminf_{n\to\infty} \int f_n \,d\mu. \]
\end{challenge}

\begin{challenge}[a summation]
% Put your code here.

\end{challenge}

\begin{challenge}[a superfluous statement]
% Put your code here.

\end{challenge}

\begin{challenge}[the harmonic series] The harmonic series diverges.
\begin{proof}
% Put your code here.

\end{proof}
\end{challenge}

\begin{challenge}[a symbol that doesn't exist]
% Put your code here.

\end{challenge}

\begin{challenge}[Knuth's up-arrow notation]
% Put your code here.

\end{challenge}

% Define the environment tfprob and possibly some lengths and counters here.


\begin{challenge}[true/false questions]
Answer the following with T or F.

\begin{tfprob}
There exists a well-order on $\prec$ on $\R$ such that for any $x \in \R$
the set $\{\,y \in \R \mid y \prec x\,\}$ is countable.
\end{tfprob}

\begin{tfprob}
Ap\'ery's constant
\[ \zeta(3) := \sum_{n=1}^\infty \frac{1}{n^3} \]
is transcendental.
\end{tfprob}

\begin{tfprob}
There exists a polynomial-time algorithm that determines if a given
graph is $3$-colorable.
\end{tfprob}
\end{challenge}

% You may add some code here.

\begin{challenge}[a popular number]
For each $n \in \N$, let $D(n)$ be the sum of all positive divisors of $n$.
What is $\sum_{n=1}^9 D(n)$?
% Put your code here.

\end{challenge}

% You may add some code here.


\begin{challenge}[countdown]
% Define renumerate and \ritem here.


A blade is aiming for the top.
\begin{renumerate}{3}
\ritem
Three!
\ritem
Two!
\ritem
One!
\ritem
Go \ldots\ shoot!
\end{renumerate}
Counting down is not the cup of tea of the other blade.
\begin{renumerate}{2}
\ritem
Three!
\ritem
Two!
\ritem
Oh, shoot.
\end{renumerate}
\end{challenge}

\begin{challenge}[a cyclic order]
In $dx\,dy$, $dy\,dz$, and $dz\,dx$, the basic $1$-forms $dx$, $dy$, and $dz$
always appear in cyclic pairs. See Figure \ref{fig:cycle}.

\begin{figure}[h]
% Put your code here.

\caption{The cyclic order of $dx$, $dy$, and $dz$.}
\label{fig:cycle}
\end{figure}
\end{challenge}

\begin{challenge}[Riemann sums]
Some Riemann sums of $f(x):=x^2/3$ over $[0,3]$ are shown in Figure
\ref{fig:riemann}.
% Define \riemann here.

\begin{figure}[h]
\riemann{5}
\riemann{10}

\riemann{20}
\riemann{40}
\caption{The Riemann sums for $n=5$, $10$, $20$, and $40$.}
\label{fig:riemann}
\end{figure}
\end{challenge}

% Add some code for the bibliography here.

\end{document}