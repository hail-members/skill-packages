\documentclass{article}

\begin{document}
\title{The North Wind and the Sun}
\author{Aesop}
\date{May 13th, 2022}
\maketitle

The North Wind and the Sun is one of Aesop's Fables (Perry Index 46).
It is type 298 (Wind and Sun) in the Aarne-Thompson folktale classification.
The moral it teaches about the superiority of persuation over force has made the story widely known. The full text is as follows.

The North Wind and the Sun were disputing which was the stronger, when a traveler came along wrapped in a warm cloak.
They agreed that the one who first succeeded in making the traveler take his cloak off should be considered stronger than the other.
Then the North Wind blew as hard as he could, but the more he blew the more closely did the traveler fold his cloak around him;
and at last the North Wind gave up the attempt. Then the Sun shined out warmly, and immediately the traveler took off his cloak.
And so the North Wind was obliged to confess that the Sun was the stronger of the two.

\end{document}