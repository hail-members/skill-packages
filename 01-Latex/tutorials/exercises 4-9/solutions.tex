\documentclass{amsart}
\usepackage{amssymb}
\usepackage{hyperref}

\DeclareMathOperator{\E}{\mathbf{E}}
\DeclareMathOperator{\PP}{\mathbf{P}}

\numberwithin{equation}{section}

\begin{document}

\title{Math Exercises}
\author{Your Name$_\circ$}
\date{\today}

\maketitle

\section{Exercise 4}

\subsection{Roots and superscripts}
The two roots of $x^2 = 2$ are $\pm \sqrt{2}$.
If $\sqrt{2} = p/q$ for some integers $p$ and $q$, then
we have $p^2 = 2q^2$, which is impossible.
A similar argument holds for $\sqrt[n]{2}$ with $n \ge 3$.

\subsection{Pythagorean theorem}
The Pythagorean theorem says $\alpha^2 + \beta^2 = \gamma^2$.
It follows that $\gamma = \sqrt{\alpha^2 + \beta^2}$.

\subsection{Fibonacci numbers}
Fibonacci numbers satisfy
\[ F_n = F_{n-1} + F_{n-2} \]
for $n=1,2,\dots$.

\subsection{Inner products}
If $\vec{x} = (x_1,\dots,x_n)$ and $\vec{y} = (y_1,\ldots,y_n)$,
then
\[ \vec{x}\cdot\vec{y} = x_1y_1 + \dots + x_ny_n. \]

\subsection{Repeating decimals}
If $0.\bar{9} < 1$, then $0 < 1-0.\bar{9}\le\epsilon$ should hold for
every $\epsilon>0$.

\subsection{Approximating sine}
We have $\sin x \approx x$ for small $x$.

\subsection{Closures}
If $A_1,\dots,A_n \subset X$, then $\overline{A_1\cup\dots\cup A_n}
= \overline{A_1}\cup\dots\cup\overline{A_n}$.

\subsection{Inverse function}
If $f \circ g$ and $g \circ f$ are both identities, then
$g$ is denoted by $f^{-1}$.

\subsection{Set builder notation}
If $R = \{x \mid x \not\in x\}$, then $R \in R$ and $R \not\in R$
are both true.


\section{Exercise 5}

\subsection{Quadratic formula}
If $a$, $b$, and $c$ are complex numbers where $a \ne 0$, the roots of
$ax^2 + bx + c = 0$ are given by
\[
x = \frac{-b \pm \sqrt{b^2 - 4ac}}{2a}.
\]

\subsection{Basel sum}
In 1734, Euler proved that
\[
\sum_{n=1}^\infty \frac{1}{n^2} = \frac{1}{1^2} + \frac{1}{2^2} + \frac{1}{3^2}
+ \cdots = \frac{\pi^2}{6}.
\]

\subsection{Euler's infinite product}
Euler's infinite product formula says
\[
\frac{\sin x}{x} = \prod_{n=1}^{\infty} \left(1 - \frac{x^2}{n^2\pi^2}\right).
\]

\subsection{Definition of derivative}
If $y = f(x)$, we have
\[ \frac{dy}{dx} = f'(x) = \lim_{h\to0} \frac{f(x+h)-f(x)}{h}. \]

\subsection{Stokes' theorem}
Let $S$ be an oriented surface, and $F$ be a $C^1$ vector field on $S$.
Then,
\[
\iint_S (\nabla \times F) \cdot dS = \int_{\partial S} F \cdot ds.
\]

\subsection{Integration by parts}
The variance of the standard normal random variable can be computed by the
following integration by parts:
\[ \int_{-\infty}^\infty x^2 \cdot \frac{e^{-x^2/2}}{\sqrt{2\pi}}\,dx
= \left[\frac{-xe^{-x^2/2}}{\sqrt{2\pi}}\right]_{x=-\infty}^{\infty}
+ \int_{-\infty}^\infty \frac{e^{-x^2/2}}{\sqrt{2\pi}}\,dx. \]



\section{Exercise 7}

\subsection{A random set}
\[ \left\{ \frac{1}{p}+\frac{1}{q} : \text{$p$ and $q$ are prime} \right\} \]

\subsection{L\'evy equivalence theorem}
If $X_1,X_2,\ldots$ are independent, then
\[ \sum_{n=1}^\infty X_n \text{ converges a.s.}\quad
\text{if and only if}\quad \sum_{n=1}^\infty X_n
\text{ converges in distribution}.
\]

\subsection{Completing the computation}
We continue the computation given in 2.6:
\[
\begin{split}
\int_{-\infty}^\infty x^2 \cdot \frac{e^{-x^2/2}}{\sqrt{2\pi}}\,dx
&= \left[\frac{-xe^{-x^2/2}}{\sqrt{2\pi}}\right]_{x=-\infty}^{\infty}
+ \int_{-\infty}^\infty \frac{e^{-x^2/2}}{\sqrt{2\pi}}\,dx. \\
&= (0-0) + 1 \\
&= 1.
\end{split}
\]

\subsection{A long inequality}
\[
\begin{split}
|\mathbf{E}[f(Z)]-\mathbf{E}[f(S)]-&\mathbf{E}[f''(S)]\mathbf{E}[Y^2]/2| \\
&\le \frac{\epsilon}{2}\mathbf{E}[Y^2]+M\mathbf{E}[Y^2;|Y|>\delta].
\end{split}
\]

\subsection{Definition of cross product}
The cross product of $\mathbf{x}=(x_1,x_2,x_3)$ and $\mathbf{y}=(y_1,y_2,y_3)$
is given by
\[ \mathbf{x}\times\mathbf{y} = \begin{vmatrix}
\mathbf{i} & \mathbf{j} & \mathbf{k} \\
x_1 & x_2 & x_3 \\
y_1 & y_2 & y_3
\end{vmatrix}. \]

\subsection{Large matrices}
\[
\begin{pmatrix}
0 & x_{12} & x_{13} & \cdots & x_{1n} \\
x_{21} & 0 & x_{23} & \cdots & x_{2n} \\
x_{31} & x_{32} & 0 & \cdots & x_{3n} \\
\vdots & \vdots & \vdots & \ddots & \vdots \\
x_{n1} & x_{n2} & x_{n3} & \cdots & 0
\end{pmatrix}
\]


\section{Exercise 8}

\theoremstyle{plain}
\newtheorem{theorem}{Theorem}[section]
\newtheorem{corollary}[theorem]{Corollary}

\theoremstyle{remark}
\newtheorem{remark}{Remark}[section]

\begin{theorem}[Baum--Katz]
If either
\begin{itemize}
\item
$t \ge 2$ and $r > t/2$; or
\item
$0 < t < 2$ and $r \ge 1$,
\end{itemize}
then
\begin{equation} \label{eq:bk1}
\E X=0\text{ (in case $t\ge1$ and $r/t\le1$)} \quad\text{and}\quad
\E|X|^t < \infty
\end{equation}
is equivalent to
\begin{equation} \label{eq:bk2}
\sum_{n=1}^\infty n^{r-2}\PP(|S_n|>n^{r/t}\epsilon) < \infty
\quad \text{for all $\epsilon > 0$}.
\end{equation}
\end{theorem}

\begin{proof}
It is trival that \eqref{eq:bk1} implies \eqref{eq:bk2}.
We leave the other direction as an exercise.
\end{proof}

\begin{theorem}[continuity] \label{thm:cont}
For any function $f \colon \mathbb{R} \to \mathbb{R}$, the following are
equivalent:
\begin{enumerate}
\item
$f$ is continuous;
\item
$f^{-1}(U)$ is open for any open $U \subset \mathbb{R}$;
\item
$f^{-1}(C)$ is closed for any closed $C \subset \mathbb{R}$.
\end{enumerate}
\end{theorem}

\begin{corollary} \label{cor:comp}
The composition of any two continuous functions from $\mathbb{R}$ to
itself is continuous.
\end{corollary}

\begin{remark}
The previous corollary holds for any continuous maps between
arbitrary topological spaces.
\end{remark}

\begin{proof}[Proof of Corollary \ref{cor:comp}]
Let $f,g \colon \mathbb{R} \to \mathbb{R}$ be continuous maps.
For any open $U \subset \mathbb{R}$, the set $f^{-1}(U)$ is open
by Theorem \ref{thm:cont}, and so is $g^{-1}(f^{-1}(U))$ by the same theorem.
Since $g^{-1}(f^{-1}(U)) = (f\circ g)^{-1}(U)$, the continuity of
$f \circ g$ follows.
\end{proof}

%\cite{BK65} \cite{Erdos49}
%
%\bibliographystyle{alpha}
%\bibliography{references.bib}
\end{document}
